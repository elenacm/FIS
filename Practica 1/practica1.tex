\documentclass[a4paper, 11pt]{article}

\usepackage{amsmath}
\usepackage[protrusion=true,expansion=true]{microtype}
\usepackage{mathpazo}
\usepackage{booktabs}
\usepackage{multicol}
\usepackage{multirow}
\usepackage[spanish]{babel}
\usepackage[latin1]{inputenc}
\usepackage{enumerate}
\usepackage{listings}
\usepackage{graphicx}
\usepackage{wrapfig}
\usepackage{longtable}

\title{\textbf{Pr\'actica 1}\\ \textbf{Lista inicial de requisitos}}
\author{Alba Aurora Moreno Peinado\\ Manuel Ortiz Hita \\ Pablo Zafra Jim\'enez \\ Elena Cantero Molina}
\date{9 de Marzo de 2018}

\begin{document}

\maketitle
{\parskip=20pt\tableofcontents}

\setlength{\parskip}{5pt}

\section{Descripci\'on del problema y objetivos}
Este proyecto quiere abordar un sistema de gesti\'on de la empresa de transportes \textbf{Env\'iamelo S.A}. que informatice el env\'io y recepci\'on de paquetes tanto internamente para la empresa como para el usuario, incluyendo  el seguimiento y gesti\'on de paquetes, c\'alculo del coste, personal necesario, optimizaci\'on del almacenado y env\'io, etc. Es decir, las relaciones entre el usuario y la empresa, y la empresa con sus empleados.\\

\setlength{\parindent}{1pt}
\textbf{Descripci\'on inicial del problema:}
\setlength{\parindent}{12pt}
\begin{itemize}
\item La empresa Env\'iamelo S.A consta de m\'ultiples oficinas repartidas por todas las capitales de Espa\~na que hacen uso de la recepci\'on de paquetes tanto de usuarios como de otras oficinas de la propia empresa.
\item Las oficinas disponen de almacenes en los que se gestionan la prioridad de los paquetes dependiendo del modo en el que se han enviado (env\'ios expr\'es, urgentes, est\'andar, etc).
\item Dentro de cada almac\'en hay un conjunto de furgonetas encargadas de recoger los env\'ios, as\'i como de repartirlos siguiendo una ruta previamente establecida.
\item Estas rutas son programadas para una recogida y env\'io, en los que el tiempo empleado para ello sea \'optimo y eficiente, pudiendo as\'i localizar los centros de servicio m\'as cercanos al domicilio del usuario.
\item La disponibilidad de los conductores depender\'a de ciertos factores: el n\'umero de paquetes  a gestionar, zona de reparto, tramos horarios, etc.
\item Estos env\'ios y recepciones pueden ser efectuados tanto en la misma oficina de la ciudad como a trav\'es de un repartidor a domicilio (lo que supondr\'ia un coste adicional).
\item El env\'io llegar\'a en un m\'aximo de 5 d\'ias laborales. Si el usuario receptor del env\'io no se encuentra en el domicilio tendr\'a la opci\'on de ir a recogerlo a la oficina m\'as cercana, en un plazo de 5 d\'ias laborables desde el d\'ia en el que se ausent\'o.
\item El importe a pagar por los usuarios depender\'a de la distancia de env\'io, el tipo, peso y tama\~no del paquete. Es posible el pago en la oficina o a trav\'es de la pagina web. 
\item Se dispondr\'a de una pagina web en la que se podr\'an efectuar los pagos, el seguimiento del env\'io, un localizador de oficinas, elecci\'on de la hora de recogida, etc.
\item Tanto para el uso y gesti\'on de los env\'ios a trav\'es de la pagina web, como para la valoraci\'on del servicio de la empresa ser\'a necesario un registro previo, salvo para comprobaciones del n\'umero del seguimiento de un paquete.
\end{itemize}

\setlength{\parindent}{1pt}
\textbf{Objetivos}:
\setlength{\parindent}{12pt}

Los objetivos principales que debe cumplir el software que se pretende desarrollar, en cierto modo, se puede pensar en ellos como requisitos de alto nivel, son los siguientes:

\setlength{\parindent}{30pt}
\textbf{OBJ-1}. El sistema almacenar\'a y gestionar\'a toda la informaci\'on relativa al env\'io y recibo de paquetes.

\textbf{OBJ-2}. El sistema generar\'a y actualizar\'a autom\'aticamente el seguimiento de los pedidos.

\textbf{OBJ-3}. Los usuarios podr\'an decidir las fechas y el lugar de recogida.

\textbf{OBJ-4}. El sistema podr\'a gestionar diversos paquetes as\'i como la ruta de transporte m\'as \'optima.

\setlength{\parindent}{1pt}
\section{Identificar Implicados}
\textbf{Entorno de usuarios}
\setlength{\parindent}{15pt}

Los usuarios directos de la aplicaci\'on a desarrollar son dos: el administrador, encargado de gestionar el sistema y de introducir toda la informaci\'on necesaria en la base de datos del mismo, y cualquier usuario que desee consultar la informaci\'on relativa al env\'io y recibo del paquete. El primero, es buen conocedor tanto del sistema inform\'atico que debe utilizar como de toda la terminolog\'ia sobre el env\'io de paquetes utilizadas por la aplicaci\'on. Entre los \'ultimos puede haber personas con distinto nivel de conocimiento en ambas \'areas, dado que en principio cualquier persona puede acceder al sistema.\\
\setlength{\parindent}{1pt}

\textbf{Resumen de implicados}\\
\begin{tabular}{@{}llll@{}}
\toprule
Nombre & Descripci\'on   & Tipo  & Responsabilidad  \\ \midrule
Empleado & \begin{tabular}[c]{@{}l@{}}Representa a  un \\ empleado\end{tabular} & \begin{tabular}[c]{@{}l@{}}Usuario\\ producto\end{tabular} & \begin{tabular}[c]{@{}l@{}}Realiza actividades de \\ gest\'ion de la empresa\\ de mensajer\'ia.\\ Atiende a los clientes.\end{tabular} \\ \midrule
Encargado & \begin{tabular}[c]{@{}l@{}}Representa al \\ due\~no o encargado\end{tabular}  & \begin{tabular}[c]{@{}l@{}}Usuario\\ producto\end{tabular} & \begin{tabular}[c]{@{}l@{}}Realiza actividades de\\ gesti\'on y econ\'omicas\\ de la mensajer\'ia.\\ Atiende a los clientes y\\ a los repartidores.\end{tabular} \\ \midrule
Repartidor & \begin{tabular}[c]{@{}l@{}}Representa al\\ repartidor\end{tabular} & \begin{tabular}[c]{@{}l@{}}Usuario\\ producto y\\ sistema\end{tabular}  & \begin{tabular}[c]{@{}l@{}}Recoge y suministra los\\ pedidos de los usuarios.\end{tabular}\\ \midrule
Cliente & \begin{tabular}[c]{@{}l@{}}Representa a un\\ cliente\end{tabular} & \begin{tabular}[c]{@{}l@{}}Usuario\\ producto y \\ sistema\end{tabular} & \begin{tabular}[c]{@{}l@{}}Env\'ia y recibe los\\ paquetes.\end{tabular} \\ \midrule
Administrador & \begin{tabular}[c]{@{}l@{}}Representa a la \\ persona que gestiona\\ el sistema\end{tabular} & \begin{tabular}[c]{@{}l@{}}Usuario\\ sistema\end{tabular} & Gestiona el sistema                                                                                                                                              \\ \bottomrule
\end{tabular}

\section{Obtenci\'on de requisitos}
\setlength{\parindent}{15pt}
\subsection{Requisitos funcionales}
Descripci\'on de los requisitos m\'as importantes a nivel de funciones que debe incluir el sistema, realizando una clasificaci\'on en categor\'ia. A cada uno de los requisitos se le ha asignado un c\'odigo y un nombre, con el fin de identificarlo f\'acilmente a lo largo de todo el proyecto.

\setlength{\parindent}{20pt}
\textbf{RF-1. Gesti\'on de administradores}. Se permitir\'a dar de alta/baja a cualquier administrador del sistema, as\'i como consultar y/o modificar cualquiera de sus datos.

\setlength{\parindent}{35pt}
\textbf{RF-1.1. Alta de administrador}. Se registrar\'a cada nuevo administrador del sistema, con sus datos correspondientes.

\textbf{RF-1.2. Baja de administrador}. Se eliminar\'a toda la informaci\'on relativa a un administrador del sistema.

\textbf{RF-1.3. Consultar datos de administrador}. Se mostrar\'an los datos relativos a un		determinado administrador.

\textbf{RF-1.4. Modificar datos de administrador}. Se podr\'an cambiar los datos almacenados de un		administrador.\\

\setlength{\parindent}{20pt}
\textbf{RF-2. Gesti\'on de paquetes}.

\setlength{\parindent}{35pt}
\textbf{RF-2.1. Env\'io}. El sistema debe almacenar informaci\'on sobre los paquetes que se pueden 	 adquirir y su precio de env\'io.

\textbf{RF-2.2. Consulta de datos}. El sistema debe permitir a los usuarios buscar y consultar la  informaci\'on sobre los paquetes. 

\textbf{RF-2.3. Modificar Paquetes}. El sistema debe de permitir modificar el destino del paquete, si el paquete ha efectuado su salida previamente, se debe de ingresar un coste adicional 	correspondiente a la nueva entrega.\\

\setlength{\parindent}{20pt}
\textbf{RF-3. Gesti\'on de clientes}. Se permitir\'a dar de alta y baja a un cliente en el sistema, as\'i como consultar y/o modificar cualquiera de sus datos.

\setlength{\parindent}{35pt}
\textbf{RF-3.1. Alta de un cliente}. Se registrar\'a un nuevo cliente en el sistema, con sus datos correspondientes.

\textbf{RF-3.2. Baja de un cliente}. Se eliminar\'a del sistema toda la informaci\'on relativa a un cliente.

\textbf{RF-3.3. Consultar datos}. Se podr\'a consultar la informaci\'on relativa a un cliente.

\textbf{RF-3.4. Modificar datos}. Se podr\'a modificar la informaci\'on relativa a un cliente.

\textbf{RF-3.5. Registro}. El sistema registrar\'a la informaci\'on de los usuarios.

\textbf{RF-3.6. Herramientas}. El sistema debe permitir que los usuarios registrados env\'ien paquetes, 	y proporcionar las herramientas para que los usuarios paguen.\\

\setlength{\parindent}{20pt}
\textbf{RF-4. Gesti\'on de rutas}. Se permitir\'a incluir una nueva ruta en el sistema antes de su afectaci\'on.

\setlength{\parindent}{35pt}
\textbf{RF-4.1. Crear Ruta}. Se registrar\'a una ruta en el sistema, con sus datos correspondientes.

\textbf{RF-4.2. Eliminar Ruta}. Se borrar\'a del sistema toda la informaci\'on relativa a una ruta.

\textbf{Rf-4.3. Consultar Ruta}. Consultar los datos de una ruta. Se mostrar\'an  los datos almacenados sobre una ruta.

\textbf{Rf-4.4. Modificar Ruta}. Se podr\'an cambiar los datos almacenados sobre una determinada 	ruta.

\setlength{\parindent}{12pt}
\subsection{Requisitos no funcionales}
Aqu\'i se incluyen algunas restricciones que afectar\'an a los requisitos anteriores.

\setlength{\parindent}{20pt}
\textbf{RNF-1. Seguridad de la informaci\'on}. Se crear\'an copias de seguridad peri\'odicas para asegurar la informaci\'on previamente almacenada ante un posible caso de error.

\textbf{RNF-2. Concurrencia}. Varios administradores podr\'an trabajar simult\'aneamente con el sistema, introduciendo informaci\'on, as\'i como varios usuarios podr\'an consultar al mismo tiempo
datos almacenados en el mismo.

\textbf{RNF-3. Autentificaci\'on}. Es necesario controlar que s\'olo las personas autorizadas (administradores) puedan acceder a las funciones de gesti\'on del sistema e introducir informaci\'on en \'el.

\textbf{RNF-4. Privacidad}. La informaci\'on proporcionada por los clientes \'unicamente ser\'a gestionada por el sistema inform\'atico, y cumplir\'a la Ley Org\'anica 15/1999, de 13 de diciembre, de Protecci\'on de Datos de Car\'acter Personal.

\textbf{RNF-5 Tiempo de respuesta}. El sistema no deber\'a de tardar m\'as de 5 segundos en mostrar los resultados de una b\'usqueda.

\subsection{Requisitos de Informaci\'on}
\setlength{\parindent}{12pt}
A continuaci\'on se indica la informaci\'on que es necesario almacenar en el sistema.
\setlength{\parindent}{20pt}

\textbf{RI-1.Administradores}.  Datos sobre las personas que gestionar\'an el sistema que se va a desarrollar.\\
Contenido: Nombre de usuario, clave de acceso, direcci\'on de correo, n\'umero de tel\'efono.\\
Requisitos asociados: RF-1.

\textbf{RI-2. Paquetes}. Informaci\'on necesaria de paquetes para darlos de altas en el sistema.\\
Contenido: C\'odigo del paquete, direcci\'on, n\'umero de albar\'an, n\'umero de seguimiento.\\
Requisitos asociados: RF-2.

\textbf{RI-3. Clientes}. Datos necesarios sobre los clientes para darlos de alta en el sistema.\\
Contenido: Nombre, dni, nacionalidad, direcci\'on y email.\\
Requisitos asociados: RF-3.

\textbf{RI-4. Rutas}. Informaci\'on a registrar sobre la ruta que debe de seguir el repartidor.\\
Contenido: Nombre de ruta, ciudades, localidad, direcciones, distancias, duraci\'on.\\
Requisitos asociados: RF-4.

\section{Glosario}

\begin{enumerate}
\item \textbf{Administrador}: Persona, probablemente con conocimientos avanzados de inform\'atica, que se encarga de gestionar y mantener el sistema en funcionamiento.
\item \textbf{Albar\'an}: Nota de entrega que firma la persona que recibe una mercanc\'ia.
\item \textbf{Clave}: C\'odigo de signos convenidos para la transmisi\'on de mensajes privados.
\item \textbf{Correo}: Servicio p\'ublico que tiene por objeto el transporte de la correspondencia oficial y privada.
\item \textbf{Env\'io}: Acci\'on de hacer que un paquete se dirija o sea llevado a alguna parte.
\item \textbf{Paquete}: Objeto que se ajusta a determinados requisitos y se envía por correo.
\item \textbf{Prioridad}: Anterioridad de algo respecto de otra cosa, en tiempo u orden.
\item \textbf{Recepci\'on}: Acci\'on y efecto de recibir un paquete.
\item \textbf{Recoger}: Acci\'on de recoger de retirar un paquete de un sitio acordado.
\item \textbf{Ruta}: Camino o direcci\'on que se toma para el prop\'osito de transmisi\'on del paquete.
\item \textbf{Seguimineto}: Acci\'on y efecto de seguir un paquete.
\item \textbf{Transporte}: Acci\'on y efecto o medio para desplazar un paquete de un lugar a otro.

\end{enumerate}

\end{document}
